\documentclass[dvipdfmx]{jsarticle}
\usepackage[hiresbb]{graphicx}
\usepackage[dvipdfmx]{hyperref}
\usepackage{pxjahyper}
\usepackage[top=10mm, bottom=20mm, left=15mm, right=15mm]{geometry}

\hypersetup{
colorlinks=true,
linkcolor=black,
urlcolor=blue
}

\author{e.wakabayashi-aa}
\title{makeについて}



\begin{document}


\maketitle

\begin{abstract}
makeは古くから使われているバッチ処理の仕組みです。
C言語のコンパイル(ビルド)に使われてきました。
目的とするファイルとそのために必要なファイルとそのために実行すべきコマンドの
三つをセットにしただけの単純な構造です。\par
アプリケーションの開発はその最終結果は1個のファイルであっても
複数のソースファイル(プログラム)と多くの中間ファイルからなり
いくつもの手順を必要とします。
当然手作業では煩雑で間違いも生じるのでバッチ処理を行うことになります。
しかしただコマンドを並べただけのバッチでは毎回全ソースをコンパイルすることになり
たった一か所の修正のたびに1時間以上も待たされることになります。
バッチに条件判定を追加するのも手ですがそれはエラーの一因ともなります。\par
makeはファイルの更新日から行うべき作業を判断し実行します。
1か所の修正のために全行程を実行するこたはありません。間違いも起こりません。
このように便利なmakeをC言語のコンパイル用途だけにとどめる手はありません。
プログラムを小さくモジュール化することにより開発が容易になりエラー時にも原因箇所特定(そこで止まる)がすぐにできます。\par
他の用途にも使えるという記述は見ますがその具体的内容はあまり見ません。
ここでその一例とmakeについての簡単な解説を紹介します。

\end{abstract}

%\tableofcontents

\newpage

\section{基礎}
まず基本的なことがら。


\subsection{makeとMakefile}
\textbf{make}は実行ファイル(UNIX系なのでWindowsの拡張子「.exe」は尽きません。)、
\textbf{Makefile}はテキストファイルです。
\textbf{Makefile}に記された内容を\textbf{make}が実行します。


\section{nmake.exe}

\textbf{nmake.exe}はWindows版のmakeです。Visual Studioに含まれます。

\subsection{有償か無償か}
詳しくいえばnmake.exeはVisual Studio のコマンドライン・ビルドツール
「Build Tools for Visual Studio 2019」の一部です。
そしてそれは無償でダウンロードできます。よってnmake.exeも無償であると判断しました。\par
\url{https://visualstudio.microsoft.com/ja/downloads/}


\subsection{どこにあるか}
かなり深いところにあります。現PCでは以下です。ここからコピーして使っています。
\begin{verbatim}
C:\Program Files (x86)\Microsoft Visual Studio\2017\BuildTools\VC\Tools\MSVC\14.15.26726\bin\Hostx64\x64
\end{verbatim}
64bit版を使いましたが32bit版もあります。他のフォルダにもあります。違いはわかりません。

\section{Makefile}
実行内容を記述したテキストファイルです。
makeと同じ場所になければなりません。ファイル名は「Makefile」でなければなりません。
(オプションを指定すればその限りではありません。)

\subsection{Description Blocks}
Makefileの基本単位です。ファイルの依存関係を記述します。記述順序は任意です。
makeがその依存関係とファイル更新日から判断してやるべきことを決めます。

\begin{quote}
\begin{verbatim}
targets... : dependents...
    commands...
\end{verbatim}
\end{quote}

\begin{itemize}
\item targetsは目的とするファイル、dependents(依存)はそのために必要なファイル、
commandsはそのために実行しなければいけないコマンドです。
\item targetsとdependentは「:」(コロン)で区切ります。
\item commandの前はタブ1個
\end{itemize}

\begin{description}
\item[重要] commandの前は必ずタブでなければなりません。「空白4個」ではいけません。makeエラーの原因がの多くがこれです。
\end{description}

\url{https://docs.microsoft.com/en-us/cpp/build/reference/description-blocks?view=vs-2017}


\subsection{Targets(ターゲット)}
複数指定することができます。空白で区切ります。
ターゲットを複数指定しるとコマンドがターゲットの数だけ実行されるような気がしますが
そうはならなかったように記憶しています。一回実行してすべてのターゲット・ファイルが作成されるとそれを検知してそのブロックはもう実行されません。

\subsubsection{Pseudotargets}
ターゲットにはファイル名を指定するのですが、存在しないファイル、
絶対に作られることのないファイルも指定することができます。
そうするとどういうことが起こるかというとそのターゲットを持つブロックは
毎回必ず実行されることになります。
そのときのターゲットはファイル名というよりは「ラベル名」といった意味合いになります。
UNIX系ではそういう用途のとき「.」(ドット)で始まる「ファイル名」をし、
そのようなターゲットを「ドット・ターゲット」と呼んでいるのですが
Windowsのnmake.exeはそれを「Pseudotargets」と称しています。

\url{https://docs.microsoft.com/en-us/cpp/build/reference/pseudotargets?view=vs-2017}


\subsection{Dependents(依存)}
ターゲットが依存するファイルです。空白で区切って複数指定できます。
ここに前述のドット・ターゲットも指定できるのですがあまりお勧めできません。
なぜならば、ドット・ターゲットは毎回必ず実行されますのでそれに「依存」
するファイルも毎回作成のためのコマンドが実行されることになります。
すでに最新版が存在するファイルの作成作業が再度行われることになり
makeを使う意味が半減します。

\url{https://docs.microsoft.com/en-us/cpp/build/reference/dependents?view=vs-2017}

\subsection{長い行}
複数ターゲットや依存を指定すると行が長くなります。
行末に「\textbackslash」(バックスラッシュ)を置くことで行の分割ができます。


\subsection{マクロ(変数)}
「マクロ」と言えばプログラムのことですが、makeが作られたのは何十年も前。
現在における「変数」のことを当時は「マクロ」と呼んでいたのでしょう。 \par
\vspace{\baselineskip}
マクロ名=文字列 \par
\vspace{\baselineskip}
マクロ名に使えるのは英数字と「\_」(アンダースコア)、大文字小文字を区別
します。マクロ名にマクロを含めることができます。
マクロの参照は「\$(マクロ名)」とします。
「\$」をマクロ文字列に含めたいときは「\$\$」とします。
\url{https://docs.microsoft.com/en-us/cpp/build/reference/defining-an-nmake-macro?view=vs-2017}

\subsubsection{定義済みマクロ}

\begin{description}
\item[\$@] 実行中のターゲット
\item[\$MAKEDIR] nmake.exeのディレクトリ
\end{description}

\subsubsection{環境変数}
環境変数もマクロのようにして参照することができます。\par
\vspace{\baselineskip}
\$(環境変数名)
\vspace{\baselineskip} \par
但し、環境変数値の中に「\$」が含まれているとそれをマクロと見なし変換しようと試みます。


\section{実行例1}

\subsection{Makefileの確認}
echoコマンドで文字列「AAA」を書き出すことにことによりa.txtファイルを
作成しえいます。
ターゲットは「a.txt」で依存はありません。
コマンドは\verb+echo AAA > a.txt+
です。
次は「b.txt」で「a.txt」をコピーして作成しています。
「c.txt」も同様です。
「.clean」はドット・ターゲットです。ファイルをクリアするときに使います。

ブロックは分かりやすいように実行順に記述していますが実際には順不同です。

\begin{quote}
\begin{verbatim}
N:\about_make\sample>type Makefile
a.txt :
        echo AAA > a.txt

b.txt : a.txt
        copy a.txt b.txt

c.txt : b.txt
        copy b.txt c.txt

.clean :
        DEL /Q a.txt b.txt c.txt

\end{verbatim}
\end{quote}


\subsection{c.txtを作成}
ターゲットとして「c.txt」を指定してnmak.exeを実行しました。\par
c.txtとその途中に作られたb.txtとa.txtが確認できます。

\begin{quote}
\begin{verbatim}
N:\about_make\sample>nmake.exe c.txt

Microsoft (R) Program Maintenance Utility Version 14.14.26433.0
Copyright (C) Microsoft Corporation.  All rights reserved.

        echo AAA > a.txt
        copy a.txt b.txt
        1 個のファイルをコピーしました。
        copy b.txt c.txt
        1 個のファイルをコピーしました。

N:\about_make\sample>
N:\about_make\sample>dir /b *.txt
a.txt
b.txt
c.txt

N:\about_make\sample>

\end{verbatim}
\end{quote}


\subsection{b.txtを削除して再実行}
「b.txt」を削除して、再度実行しています。\par
「/nologo」オプションでロゴを非表示にしています。\par
a.txtはすでに存在しているのでコマンドは実行されません。\par
c.txtも存在するのですが更新日が依存しているb.txtより古いので
コマンドが実行されました。
\begin{quote}
\begin{verbatim}
N:\about_make\sample>del b.txt

N:\about_make\sample>dir /b *.txt
a.txt
c.txt

N:\about_make\sample>nmake.exe /nologo c.txt
        copy a.txt b.txt
        1 個のファイルをコピーしました。
        copy b.txt c.txt
        1 個のファイルをコピーしました。

N:\about_make\sample>dir /b *.txt
a.txt
b.txt
c.txt

N:\about_make\sample>
\end{verbatim}
\end{quote}


\subsection{.clean}
ターゲットに「.clean」を指定して実行しました。
a.txt, b.txt, c.txt が削除されました。

\begin{quote}
\begin{verbatim}
N:\about_make\sample>dir /b *.txt
a.txt
b.txt
c.txt

N:\about_make\sample>nmake.exe /nologo .clean
        DEL /Q a.txt b.txt c.txt

N:\about_make\sample>dir /b *.txt
ファイルが見つかりません

N:\about_make\sample>

\end{verbatim}
\end{quote}

\subsection{コマンドの確認}
「/n」オプションを付けて実行すると、コマンドは実行されず表示されるだけです。\par
その他のオプションは「/?」オプションで確認できます。

\begin{quote}
\begin{verbatim}
N:\about_make\sample>nmake.exe /nologo c.txt /n
        echo AAA > a.txt
        copy a.txt b.txt
        copy b.txt c.txt
        
\end{verbatim}
\end{quote}


\section{実行例2}

実際にベア予測マクロプログラムで使用したMakefile。

\subsection{コード}

\begin{verbatim}
     1	
     2	LSPP   = \\bsu01119\WKB\LS-PrePost-4.5-x64\lsprepost4.5_x64.exe
     3	RS     = \\bsu01119\WKB\R\R-3.5.3\bin\x64\Rscript.exe
     4	PANDOC = \\bsu01119\WKB\Pandoc\pandoc.exe
     5	
     6	
     7	.usage :
     8		@type <<
     9	Usage:
    10	   nmake.exe .clean                      workフォルダ内消去
    11	   nmake.exe  work\log.yaml
    12	   nmake.exe  work\intfor_nodelist.csv   ノード・リスト作成
    13	   nmake.exe .chk_nodelist               ノード・リスト確認
    14	   nmake.exe  work\mold_node_orderd.csv  モールド・ノード・リスト作成
    15	   nmake.exe .chk_moldnode               モールド・ノード・リスト確認
    16	   nmake.exe  work\report.html           結果レポート作成
    17	   nmake.exe  work\press12vh.csv         圧力グラフ作成
    18	   nmake.exe .save                       workフォルダをresultsフォルダにコピー
    19	
    20	<<
    21	
    22	
    23	.clean:
    24		del /Q work\*
    25		IF EXIST work\report_files RD /S /Q work\report_files
    26		echo  dummy  >  work\.gitkeep
    27	
    28	
    29	#.chkvhline :
    30	#	powershell.exe -File prog\check_vhline.ps1
    31	
    32	
    33	.chk_nodelist : work\intfor_nodelist.csv work\element.dyn
    34		@echo #
    35		@echo #    ノード並びの確認
    36		@echo #
    37		$(RS) prog\checkorder.R work\intfor_nodelist.csv work\element.dyn
    38	
    39	
    40	.chk_moldnode : work\mold_node_orderd_201.csv \
    41	                work\mold_node_orderd_202.csv \
    42	                work\mold_node_orderd_203.csv \
    43					work\mold_elem.dyn
    44		@echo #
    45		@echo #    モールド・ノード並びの確認
    46		@echo #
    47		$(RS) prog\checkorder.R work\mold_node_orderd_201.csv work\mold_elem.dyn
    48		$(RS) prog\checkorder.R work\mold_node_orderd_202.csv work\mold_elem.dyn
    49		$(RS) prog\checkorder.R work\mold_node_orderd_203.csv work\mold_elem.dyn
    50	
    51	
    52	.save :
    53		@echo #
    54		@echo #    Copy work to results
    55		@echo #
    56		powershell.exe -File prog\copy_work_to_result.ps1 work results
    57	
    58	
    59	#
    60	# 個別ファイル
    61	#
    62	
    63	work\log.yaml     :
    64		@echo #
    65		@echo #    $@
    66		@echo #
    67		powershell.exe -File prog\logyaml.ps1
    68	
    69	
    70	work\VH_line.cfile : work\log.yaml
    71		@echo #
    72		@echo #    $@
    73		@echo #
    74		powershell.exe -File prog\vhline.ps1
    75	
    76	
    77	work\edgenodes.csv : work\element.dyn \
    78	                     work\node.dyn
    79		@echo #
    80		@echo #    $@
    81		@echo #
    82		$(RS) prog\get_edgenodes5.R  work\element.dyn \
    83	                                       work\node.dyn    \
    84	                                       work\edgenodes.csv
    85	
    86	
    87	work\intfor_nodelist.csv : work\VH_line.cfile \
    88	                           work\edgenodes.csv \
    89	                           work\node_last.dyn
    90		@echo #
    91		@echo #    $@
    92		@echo #
    93		$(RS) prog\trim_nodes.R work\VH_line.cfile \
    94	                                  work\edgenodes.csv \
    95	                                  work\node_last.dyn \
    96	                                  work\intfor_nodelist.csv
    97	
    98	
    99	work\debug_disp_edge.cfile : work\edgenodes.csv
   100		@echo #
   101		@echo #    $@
   102		@echo #
   103		powershell.exe -File prog\cfile_for_disp_nodes.ps1 work\edgenodes.csv work\debug_disp_edge.cfile
   104	
   105	
   106	work\debug_disp_intfor.cfile : work\intfor_nodelist.csv
   107		@echo #
   108		@echo #    $@
   109		@echo #
   110		powershell.exe -File prog\cfile_for_disp_nodes.ps1 work\intfor_nodelist.csv work\debug_disp_intfor.cfile
   111	
   112	
   113	work\debug_area_node.cfile : work\area_node.csv
   114		@echo #
   115		@echo #    $@
   116		@echo #
   117		powershell.exe -File prog\cfile_for_area_node.ps1 work\area_node.csv work\debug_area_node.cfile
   118	
   119	
   120	work\debug_intfor.png : work\node_last.dyn \
   121	                        work\edgenodes.csv \
   122	                        work\intfor_nodelist.csv  \
   123							work\VH_line.cfile
   124		@echo #
   125		@echo #    $@
   126		@echo #
   127		$(RS)  prog\debug_intfor.R  work\node_last.dyn        \
   128	                              work\edgenodes.csv        \
   129	                              work\intfor_nodelist.csv  \
   130	                              work\VH_line.cfile        \
   131	                              work\debug_intfor.png
   132	
   133	
   134	work\intfor1.cfile : work\intfor_nodelist.csv work\log.yaml
   135		@echo #
   136		@echo #    $@
   137		@echo #
   138		powershell.exe -File prog\intfor1_cfile.ps1
   139	
   140	
   141	work\intfor2.cfile : work\intfor_nodelist.csv work\log.yaml
   142		@echo #
   143		@echo #    $@
   144		@echo #
   145		powershell.exe -File prog\intfor2_cfile.ps1
   146	
   147	
   148	work\getdyn.cfile : work\log.yaml
   149		@echo #
   150		@echo #    $@
   151		@echo #
   152		powershell.exe -File prog\getdyn.ps1
   153	
   154	
   155	work\element.dyn        \
   156	work\node.dyn           \
   157	work\node_last.dyn      \
   158	work\mold_elem.dyn      \
   159	work\mold_node.dyn      \
   160	work\mold_node_last.dyn : work\getdyn.cfile
   161		@echo #
   162		@echo #    $@
   163		@echo #
   164		$(LSPP) c=$(MAKEDIR)\work\getdyn.cfile -nographics
   165	
   166	
   167	work\mold_node_orderd_201.csv : work\mold_elem.dyn \
   168	                                work\mold_node.dyn
   169		@echo #
   170		@echo #    $@
   171		@echo #
   172		$(RS) prog\order_mold_node.R "201" work\mold_elem.dyn work\mold_node.dyn work\mold_node_orderd_201.csv
   173	
   174	
   175	work\mold_node_orderd_202.csv : work\mold_elem.dyn \
   176	                                work\mold_node.dyn
   177		@echo #
   178		@echo #    $@
   179		@echo #
   180		$(RS) prog\order_mold_node.R "202" work\mold_elem.dyn work\mold_node.dyn work\mold_node_orderd_202.csv
   181	
   182	
   183	work\mold_node_orderd_203.csv : work\mold_elem.dyn \
   184	                                work\mold_node.dyn
   185		@echo #
   186		@echo #    $@
   187		@echo #
   188		$(RS) prog\order_mold_node.R "203" work\mold_elem.dyn work\mold_node.dyn work\mold_node_orderd_203.csv
   189	
   190	
   191	work\mold_node_orderd.csv : work\mold_node_orderd_201.csv \
   192	                            work\mold_node_orderd_202.csv \
   193								work\mold_node_orderd_203.csv
   194		@echo #
   195		@echo #    $@
   196		@echo #
   197		powershell.exe -File prog\order_mold_node.ps1 work\mold_node_orderd_201.csv \
   198	                                                  work\mold_node_orderd_202.csv \
   199	                                                  work\mold_node_orderd_203.csv \
   200	                                                  work\mold_node_orderd.csv
   201	
   202	
   203	work\vhnode.csv : work\mold_node_last.dyn      \
   204	                  work\node_last.dyn           \
   205	                  work\mold_node_orderd.csv    \
   206	                  work\intfor_nodelist.csv     \
   207	                  work\VH_line.cfile
   208		@echo #
   209		@echo #    $@
   210		@echo #
   211		$(RS)  prog\vhnode.R  work\mold_node_last.dyn      \
   212	                                work\node_last.dyn           \
   213	                                work\mold_node_orderd.csv    \
   214	                                work\intfor_nodelist.csv     \
   215	                                work\VH_line.cfile           \
   216	                                work\vhnode.csv
   217	
   218	
   219	work\range.yaml : work\vhnode.csv work\intfor_nodelist.csv work\mold_node_orderd.csv
   220		@echo #
   221		@echo #    $@
   222		@echo #
   223		$(RS)  prog\range.R  work\vhnode.csv  work\intfor_nodelist.csv  work\mold_node_orderd.csv  work\range.yaml
   224	
   225	
   226	work\press12.csv : work\intfor1.cfile work\intfor2.cfile
   227		@echo #
   228		@echo #    $@
   229		@echo #
   230		start       $(LSPP)  c=$(MAKEDIR)\work\intfor1.cfile -nographics
   231		start /WAIT $(LSPP)  c=$(MAKEDIR)\work\intfor2.cfile -nographics
   232		$(RS) prog\press_merge.R work\pressure1.csv  work\pressure2.csv  work\press12.csv
   233	
   234	
   235	work\press12vh.csv : work\press12.csv          \
   236	                     work\VH_line.cfile        \
   237	                     work\intfor_nodelist.csv
   238		@echo #
   239		@echo #    $@
   240		@echo #
   241		$(RS)  prog\press_insert_vhline.R work\press12.csv  work\vh.csv  work\press12vh.csv
   242	
   243	
   244	work\vh.csv : work\press12.csv  work\VH_line.cfile  work\intfor_nodelist.csv work\node_last.dyn
   245		@echo #
   246		@echo #    $@
   247		@echo #
   248		$(RS)  prog\vh.R  work\press12.csv  work\VH_line.cfile  work\intfor_nodelist.csv  work\node_last.dyn  work\vh.csv
   249	
   250	
   251	work\press.pgm : work\press12.csv  work\vh.csv
   252		@echo #
   253		@echo #    $@
   254		@echo #
   255		$(RS)  prog\press_pgm.R  work\press12.csv  work\vh.csv  work\press.pgm
   256	
   257	
   258	work\press.png : work\press.pgm
   259		@echo #
   260		@echo #    $@
   261		@echo #
   262		magick.exe  convert  -scale  "500%x50%"  work\press.pgm  work\press.png
   263	
   264	
   265	work\tire_mold_states.cfile : work\vh.csv work\log.yaml
   266		@echo #
   267		@echo #    $@
   268		@echo #
   269		powershell.exe -File prog\tire_mold_states_cfile.ps1
   270	
   271	
   272	work\tire_mold_states_list.csv : work\tire_mold_states.cfile
   273		@echo #
   274		@echo #    $@, tire_state_xxxx.dyn, mold_state_xxxx.dyn
   275		@echo #
   276		powershell.exe -File prog\tire_mold_states_list.ps1
   277	
   278	
   279	work\area2.csv : work\range.yaml               \
   280	                 work\tire_mold_states_list.csv
   281		@echo #
   282		@echo #    $@
   283		@echo #
   284		powershell.exe  prog\area2.ps1
   285	
   286	
   287	work\areavh.csv : work\vh.csv work\log.yaml work\area2.csv
   288		@echo #
   289		@echo #    $@
   290		@echo #
   291		$(RS)  prog\areavh.R
   292	
   293	
   294	work\farthest_list.csv : work\range.yaml
   295		@echo #
   296		@echo #    $@
   297		@echo #
   298		powershell.exe -File prog\farthest.ps1
   299	
   300	
   301	work\farthest.csv : work\farthest_list.csv
   302		@echo #
   303		@echo #    $@
   304		@echo #
   305		powershell.exe -File prog\farthest_bind.ps1
   306	
   307	
   308	work\area_node.csv : work\range.yaml
   309		@echo #
   310		@echo #    $@
   311		@echo #
   312		$(RS) --encoding="UTF-8"  prog\area_node.R  work\range.yaml  work\area_node.csv
   313	
   314	
   315	work\report.html : work\areavh.csv       \
   316	                   work\area_node.csv    \
   317					   work\vh.csv           \
   318					   work\farthest.csv     \
   319	                   work\log.yaml
   320		@echo #
   321		@echo #    range_sdtate_xxxx.png
   322		@echo #
   323		powershell.exe -File prog\rangepng.ps1 work\vh.csv
   324		@echo #
   325		@echo #    $@
   326		@echo #
   327		$(RS) <<
   328	setwd('work')
   329	knitr::knit(input='../prog/report.Rmd', output='report.md', encoding='UTF-8')
   330	<<
   331		$(PANDOC) -t html -o work\report.html -M charset:UTF-8 -M pagetitle:BareY2 -s work\report.md
   332	
   333	
   334	
   335	

\end{verbatim}

\subsection{解説}

\begin{verbatim}
     2	LSPP   = \\bsu01119\WKB\LS-PrePost-4.5-x64\lsprepost4.5_x64.exe
     3	RS     = \\bsu01119\WKB\R\R-3.5.3\bin\x64\Rscript.exe
     4	PANDOC = \\bsu01119\WKB\Pandoc\pandoc.exe
\end{verbatim}

LS-PREPOSTなどの実行に必要なアプリケーションは通常はローカルPCに
インストールしておくのだが、共有サーバーに置くことによってインストールを不要に
している。
「=」の両端に空白がありクオーティングも行っていないが、マクロ値に空白を含む場合は
こういうやり方はできない。

\begin{verbatim}
     7	.usage :
\end{verbatim}

「.usage」ターゲットで簡単な使い方を表示するようにしている。
makeを引数なし(ターゲットを指定しないで)で実行すると
Makefileに書かれた最初のターゲットが実行される。

\begin{verbatim}
     8		@type <<
     ...
    20	<<
\end{verbatim}

インライン・ファイル(ヒア・ドキュメント)を使用しています。
コマンド名に「@」を付けるとそのコマンドは実行時に表示されません。

\url{https://docs.microsoft.com/en-us/cpp/build/reference/specifying-an-inline-file?view=vs-2017}


\begin{verbatim}
    63	work\log.yaml     :
    64		@echo #
    65		@echo #    $@
    66		@echo #
    67		powershell.exe -File prog\logyaml.ps1
\end{verbatim}
「\$@」は現在のターゲットを表すマクロです。
現在どんな作業(作ろうとしているファイル名)をしているかを表示しています。

\begin{verbatim}
\end{verbatim}




\end{document}
